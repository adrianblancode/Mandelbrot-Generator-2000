% !TEX TS-program = pdflatex
% !TEX encoding = UTF-8 Unicode

% This is a simple template for a LaTeX document using the "article" class.
% See "book", "report", "letter" for other types of document.
\batchmode 
\documentclass[11pt]{article} % use larger type; default would be 10pt

\usepackage[utf8]{inputenc} % set input encoding (not needed with XeLaTeX)

%%% Examples of Article customizations
% These packages are optional, depending whether you want the features they provide.
% See the LaTeX Companion or other references for full information.

%%% PAGE DIMENSIONS
\usepackage{geometry} % to change the page dimensions
\geometry{a4paper} % or letterpaper (US) or a5paper or....
% \geometry{margin=2in} % for example, change the margins to 2 inches all round
% \geometry{landscape} % set up the page for landscape
%   read geometry.pdf for detailed page layout information

\usepackage{graphicx} % support the \includegraphics command and options

% \usepackage[parfill]{parskip} % Activate to begin paragraphs with an empty line rather than an indent

%%% PACKAGES
\usepackage{booktabs} % for much better looking tables
\usepackage{array} % for better arrays (eg matrices) in maths
\usepackage{paralist} % very flexible & customisable lists (eg. enumerate/itemize, etc.)
\usepackage{verbatim} % adds environment for commenting out blocks of text & for better verbatim
\usepackage{subfig} % make it possible to include more than one captioned figure/table in a single float
% These packages are all incorporated in the memoir class to one degree or another...

%%% HEADERS & FOOTERS
\usepackage{fancyhdr} % This should be set AFTER setting up the page geometry
\pagestyle{fancy} % options: empty , plain , fancy
\renewcommand{\headrulewidth}{0pt} % customise the layout...
\lhead{}\chead{}\rhead{}
\lfoot{}\cfoot{\thepage}\rfoot{}

%%% SECTION TITLE APPEARANCE
\usepackage{sectsty}
\allsectionsfont{\sffamily\mdseries\upshape} % (See the fntguide.pdf for font help)
% (This matches ConTeXt defaults)

%%% ToC (table of contents) APPEARANCE
\usepackage[nottoc,notlof,notlot]{tocbibind} % Put the bibliography in the ToC
\usepackage[titles,subfigure]{tocloft} % Alter the style of the Table of Contents
\renewcommand{\cftsecfont}{\rmfamily\mdseries\upshape}
\renewcommand{\cftsecpagefont}{\rmfamily\mdseries\upshape} % No bold!

%%% END Article customizations

%%% The "real" document content comes below...

\title{Slutrapport}
\author{Adrian Blanco och Casper Winsnes}
%\date{} % Activate to display a given date or no date (if empty),
         % otherwise the current date is printed 

\begin{document}
\maketitle

\section{Programbeskrivning. Beskriv detaljerat vad programmet gör:}

{\bf A:}

Vårt program är en mandelbrotgenerator. Programmet kan generera bilder av mandelbrotset samt juliaset, och förhoppningsvis i framtiden andra användardefinierade funktioner. Programmet ska ha ett enkelt och minimalistiskt GUI med få knappar, i fokus ska de genererade bilderna ligga.
Användare ska kunna bestämma graden av antialiasing, zoom, panorering och ha möjlighet att spara den generade bilden som en bildfil.
Programmet ska dessutom ha fullskärmsfunktionalitet och fönsterfunktionalitet. Med hjälp av biblioteket Aparapi kan vi få våran javakod att konverteras till openCL och köras i grafikkortet för extra hastighet.
\\
\\
{\bf B:} Det som finns på skärmen är en stor bild med ett mandelbrotset med ett fåtal knappar i hörnet som sköter olika inställningar så som fullskärm/fönster och spara bild. 

Programmet har ett förhållandevis minimalistiskt GUI med fokus på bilderna. Alla menyknappar ligger i hörnet för att inte vara ivägen. På mac finns knapparna längst upp istället då mac implementerar knappar på ett annorlunda sätt än det vanliga.

De viktigaste inställningarna, antialiasingalternativ med mer, går att göra i en \\inställningsskärm som dyker upp då man trycker på den knapp som leder dit. I denna kan man växla mellan mandelbrotset och juliaset, vilka man måste definiera själv.

Programmet har fullskärmsfunktionalitet, men det fungerar lite olika på mac/annat på grund av hur macs skärmar fungerar.

Det finns funktionalitet att köra all bildgenerering direkt på grafikkortet, dock bara om man har ett grafikkort som stöder detta.
\\
\\
{\bf C:} Den största skillnaden är att vi inte tog med några fördefinierade funktioner förutom det grundläggande mandelbrotsetet.

I övrigt har inte allt för mycket ändrats.

\section{Användarbeskrivning. Vem kommer att använda ert program? Vilka antaganden gör ni om användarna? Är de vana datoranvändare, är de specialister, nybörjare, små barn?}

{\bf A:}

Användare ska kunna vara vem som helst, då fokus ska ligga på enkelhet i framställningen. De användargenerade bilderna med egna funktioner lär dock antagligen kräva viss kunskap om mandelbrotset för att kunna användas fullt ut. Viss kunskap om inställningsmöjligheterna (framförallt antialiasing) kan behöva förklaras för en ovanare användare.
\\
\\
{\bf B:} 
\noindent Grundläggande delar av programmet går att hantera av vem som helst (möjlighet att zooma och spara bilder). 

\noindent Det finns inga intruktioner vilket skulle kunna orsaka problem för en helt ovan användare.

\noindent I princip alla inställningar i inställningspanelen kräver en del förkunskaper om vad orden betyder.
\\
\\
{\bf C:} Programmet blev inte riktigt så enkelt som det var tänkt. Det har gjort att det antagligen behöver implementeras intruktioenr i framtida versioner.

Vem som helst kan använda de grundläggande delarna av programmet enkelt, speciellt om man har viss tidigare datorvana, men det är inte helt uppenbart för alla vad som går att göra.

Egna juliaset kräver definitivt förkunskaper eftersom det inte finns någon förklaring för vad detta betyder i programmet.

\section{Användarscenarier. Ge minst två exempel på scenarier där en av era tänkta användare använder programmet. Beskriv i detalj vad de ser, vilken typ av input de måste ge, hur de ger sin input och hur programmets output ser ut.}

{\bf A:}

1. Tänkt användare: tonåring/20-årsåldern med intresse för matematik. Har studerat fraktaler tidigare.
Användaren vill se hur ett mandelbrotset ser ut och ha möjligheten att fokusera på de bitar som hen finner intressanta. Användaren vill även undersöka mer avancerade funktioner (exempelvis juliaset).
Det första som användaren ser är det förinställda mandelbrotset som genereras vid startandet av programmet. Det enda input som krävs är musklickningar för att kunna zooma. För att spara bilder och ändra inställningar finns det självförklarande knappar att klicka på i ena hörnet, en av knapparna öppnar en meny för avancerade inställningar som sedan användaren kan välja mellan för att ändra mandelbrotsettet. För att generera mer avancerade funktioner finns några förinställda i inställningsmenyn samt en ruta för personligt input, funktionalitet för dessa är detsamma som för mandelbrotset.
Programmets output är en mandelbrotfraktal som visas i bild. Bildfiler är ett valbart output om användaren vill. \\

2. Tänkt användare: lågstadiebarn med viss datorvana med lärare/förälder vid sidan om.
Lärare/förälder vill att barnet ska få se fina bilder och mönster och står bredvid för att hjälpa till vid eventuella problem/oförståelse.
Input är väldigt enkelt, varför barnet instinktivt borde klara av att hantera de grundläggande delarna av programmet bara genom att klicka sig fram. Om barnet vill ändra inställningar med mer kan det dock behöva hjälp av en äldre person som kan hjälpa det att förstå vad de olika inställningarna betyder. 
\\
\\
{\bf B:}
Våra testgrupper var:
\begin{enumerate}
\item Datateknikstuderande från KTH.
\item Mellanstadiebarn med viss datorvana stod tillsammans med ungdom som lärt sig hantera programmet förut. 
Barnen hade aldrig sett mandelbrotset förut och lät intresserad då vi beskrev det lite kort för det.
\end{enumerate}
{\bf C:} Testgrupperna stämde ungefärligt överens med vad vi hade tänkt oss. Barnen var lite äldre än tänkt men det orsakade antagligen inte någon större skillnad.

\section{Testplan. Beskriv hur ni tänker testa programmet. I den här uppgiften ska ni lägga extra vikt vid användartestning. Beskriv vilka uppgifter som testanvändaren ska utföra. De två användarscenarierna ska ingå i användartestningen.}

{\bf A:}

Scenario 1: (Tänkt användare nr. 1) Programmet ska testas genom att låta användaren får en kort beskrivning om vad programmet kan göra och överlåts sedan till att testa fritt. 
Några uppgifter ska genomföras av användaren:
- Zooma
- Ändra antialiasing
- Ändra typ av bildgenerering
- Spara en bild \\

Scenario 2: (Tänkt användare nr. 2) Programmet ska testas genom att vi visar användaren ett mandelbrotset (i programmet). Vi ber sedan användaren att gissa sig till hur man zoomar in på en punkt i programmet och ber henom även att zooma ut. Användaren ska även testa de mer avancerade funktionerna men då med viss assisstans från en mer erfaren användare av programmet.
\\
\\
{\bf B:} 
1. \emph{Datateknikstuderande.} Användarna förstod någorlunda bra. Den mest oförstådda funktionen var juliaseten som vi sedan tidigare antagit krävde bra förkunskaper. Det största klagomålet var att UI:t inte var i en separat tråd vilket orsakar att programmet låser sig under zoom.

Dessa användare klarade allt vi hade tänkt oss: zoom, antialiasing, bildsparning och ändring av bildgenerering (dock med viss handledning).

2.\emph{Mellanstadiebarn med viss datorvana stod tillsammans med ungdom som lärt sig hantera programmet förut.}
Vi bad den äldre att inte lägga sig i för mycket utan låta mellanstadiebarnet sköta sig självt fram tills det att det är uppenbart att det inte förstår längre.
Barnen klarade av att hantera zoom, fullskärm och spara bilder utan hjälp men förstod inte inställningspanelen alls. 
Efter viss förklaring klarade de av att byta färger i bilden, men förstod inte koncept som antialiasing eller juliaset.
Totalt sett, lyckat test ty de klarade det som vi satt upp som mål för den målgruppen.\\

\noindent I stort sett var gick användartestningen som vi hade tänkt oss vilket innebär att vi hade tänkt rätt vad gäller stora delar av UI:t. Saker som behöver förbättring är dock möjligheten till att få tydliga instruktioner, vilket alla användare ville ha implementerat.
Förslagsvis så skulle vi kunna lägga in en "How to" som öppnas i samband med att man startar programmet.
\\
\\
{\bf C:} Testningen var ungefär enligt scenariet. Vi bad dem att genomföra en sak och de klarade det i de flesta fall intuitivt. 

Juliaset var dock svårt även för vana datoranvändare varför det behövde förklaras lite grann innan.

\section{Programdesign. Beskriv de grundläggande klasserna som ni avser att implementera och ge en beskrivning av de viktigaste metoderna i varje klass.}

{\bf A:}

MandelbrotFrame: Klassen ärver från javax.swing.JFrame och använder sig således främst av dess ärvda funktioner. Denna klass tar hand om skärmen och ska implemenetera MouseListener för att hålla koll på användarens musklick samt andra interaktioner med skärmen. Av dess metoder kommer främst MouseClicked att användas. Klassen innehåller också setFullscreen vilken gör om fönstret så att den fyller hela skärmen. \\

MandelbrotCanvas: Klassen ärver från java.awt.Canvas och använder sig främst av metoderna paint, zoom och render. Dessa använder sig av en BufferedImage att generera bilder på och sedan ritas ut på Canvasen. Klassen är tänkt att ritas ut på vår MandelbrotFrame. De mest använda funktionerna är not zoomIn() och zoomOut() som hanterar zoomningen. \\

MandelbrotGenerator: Detta är en hjälpklass som sammanfattar funktioner för GPUKernel och Antialiasingkernel eftersom dessa inte är tänkta att användaren ska interagera med. De mest använda funktionerna är nog calculate() som säger åt GPUKernel och AntialiasingKernel att arbeta, samt setAntialiasing som genom räknar ut vilka värden som ska ändras i båda kernels för att hantera ett nytt värde på antialiasing och changeSize() som används för att anpassa värdena i båda kernels till en ny fönsterstorlek. \\

GPUKernel: GPUKernel är en klass som ärver från com.amd.aparapi.Kernel och genererar en mandelbrotbild som (oftast) är större än skärmen för att kunna applicera antialiasing på den. Huvudalgoritmen sitter i run() metoden, och vi använder biblioteket aparapi som konverterar all kod i den metoden till OpenCL och exekverar den i grafikkortet (eller ifall den inte kan det så exekverar den algoritmerna i processorn i multipla trådar). Eftersom vi konverterar koden till OpenCL i run() är det väldigt många restriktioner som gäller, till exempel får man inte använda objekt eller använda funktioner som finns utanför run() i de flesta fall vilket gör att det blir mycket kod och variabelduplicering, däremot är det värt det eftersom koden går mycket snabbare. Förutom run() så är några använda funktioner setMagnification() som ändrar zoomen i bilden och erase() som ritar bilden helt svart. \\

AntialiasingKernel: AntialiasingKernel är mer eller mindre en kopia av GPUKernel förutom själva run() metoden. run() i GPUKernel genererar en stor mandelbrotbild, medan run() i AntialiasingKernel har som funktion att applicera antialiasing genom att ta medelvärdet av färgerna i vissa områden i bilden och sedan spara dem i en mindre bild. Förutom run() så är nog den mest använda funktionen setSource som berättar för klassen vart bilden som den ska applicera antialiasing på finns. \\
\\
\\
{\bf B:} 
\noindent GPUKernel: Samma som i A.\\

\noindent AntialiasingKernel: Samma som i A.\\

\noindent MandelbrotGenerator: Samma som i A. \\

\noindent MandelbrotCanvas: Utöver det som står i A finns det även en demometod som är tänkt att visa vad klassen kan göra.\\

MandelbrotFrame: Klassen ärver från javax.swing.JFrame och använder sig av dess ärvda funktioner. Den implementerar även java.awt.event.ActionListener samt \\java.awt.event.MouseListener för att känna av när användaren trycker på olika knappar och känna av var användaren klickar. Klassens främsta användning är för att skapa en skärm att visa mandelbrotseten. Eventhanteringen ser till att klassen kallar på de andra klasser som behöver känna till vad som har skett. I eventhanteringen finns även koden för att spara bilder. Det finns dessutom set- och isFullscreen() metoder som ställer in storleken på fönstret och innehållet efter att användaren valt helskärmsläge.\\

SettingsFrame: Klassen ärver från javax.swing.JFrame och implementerar \\java.awt.event.ActionListener. Klassen skapar alla knappar, sliders och rutor som behövs för användarinställningar. Då användaren trycker på "refresh" eller "refresh and close" uppdaterar den innehållet i huvudskärmen (en MandelbrotFrame) så att innehållet ritas ut med de nya inställningarna. Klassen är en inre klass till MandelbrotFrame för att kunna få tillgång till innehållet i MandelbrotFrame för att kunna genomföra alla nödvändiga förändringar.\\

OSValidator: Denna klass uppkom som en följd av att Mac OSX inte visar JComponents på samma sätt som de andra operativsystemen varför vi måste veta om programmet körs på mac eller inte. Klassen har en massa statiska metoder som returnerar true eller false baserat på datorns operativsystem.\\

TopRightCornerLayout: Klassen ärver från java.awt.LayoutManager och gör så att alla komponenter i en Container läggs ut på det sätt vi tänkt oss. Det finns en huvudkomponent vilken läggs ut över hela Containern och flera mindre komponenter som läggs ut från övre högra hörnet och längre mot vänster ju fler komponenter som läggs till.\\
\\

{\bf C:} MandelbrotFrame fick mycket mer funktionalitet än vad som urpsrunligen var tänkt. SettingsFrame tillkom också som en följd av att det blev för mycket att göra allt i MandelbrotFrame.

Vi förväntade oss inte att MacOSX skulle fungera så annorlunda som det gjorde varför OSValidator blev en nödvändighet.

TopRightCornerLayout hade vi ursprungligen inte tänkt göra men vi ansåg att det blev lättare att hantera allt UI på det viset.

\section{Tekniska frågor. En lista av tekniska frågor som ni måste hantera när ni bygger ert system. Var så detaljerad som möjligt. Ett viktigt steg mot en god design är att få ner så många frågor som möjligt på papper på ett organiserat sätt med så många förslag till lösningar som möjligt.}

{\bf A:}

\noindent Bra algoritmer:

\noindent En stor del av vårt program bygger på matematiska algoritmer, och det krävs väldigt mycket för att få dem rätt. Framför allt så är algoritmerna som tar hand om mandelbrotsettet samt antialiasingen väldigt viktiga för programmet och därför är det mycket viktigt att inte bara få dem rätt utan att också göra dem bra eftersom de kommer itereras upp emot några miljoner gånger per sekund. 
I många fall blir det en kompromiss mellan kvalitet och snabbhet och det är någonting man måste ta hänsyn till \\

\noindent GPU-acceleration:

\noindent Att få grafikkortet att räkna är inte lätt. Det finns mycket problem med trådar, väldigt anonyma crasher, restriktiva kodningsregler, och specifika hårdvarubuggar som man inte alltid kan förutse överhuvudtaget.
\\

\noindent GUI

\noindent Trots att GUIt inte kommer vara i fokus måste det ändå vara användarvänligt och inte förvirrande, ifall våra planer på GUIt uppfyller det återstår att se.
\\
\\
{\bf B:}

Vi hade stora problem med att ifall man hade en kombination av Nvidia-grafikkort och windows 7/vista så låste sig programmet (och i värsta fall hela datorn) ifall renderingen tog över 2000ms, vi löste detta genom att helt enkelt avaktivera grafikkortsaccelerationen ifall detaljen eller antialiasingen översteg de grundläggande inställningarna. Vi antog att ifall användaren behöll de grundläggande grafikinställningarna skulle inte renderingstiden överstiga 1000ms även i fullskärm, och eftersom grafikkortsaccelerationen kräver stöd för 64-bitars flyttalsprecision fungerar det inte på äldre grafikkort och därför finns det minimal risk för att renderingstiden skulle överstiga 2000ms. \\
En möjlighet att lösa problemet var att dela upp paralelliseringen i mindre delar, men detta medförde att paralelliseringen blev väldigt ineffektiv och att det helt enkelt var lättare och snabbare att använda de gamla algoritmerna i CPUn istället.\\

GUI:t är minimalistiskt men ändå rätt så användarvänligt. Symbolerna på knapparna är förhållandevis enkla att förstå, men den avancerade funktionaliteten är lite för otydligt för att kunna användas på ett enkelt och lättillgängligt sätt för en ovan användare.

{\bf C:} Skillnaden mellan våra ursprungsalgoritmer och de nuvarande är inte så enorma. De viktigaste skillnaderna är att programmet känner av om datorn klarar av att hantera GPU-acceleration.

\noindent GUI:t blev som vi hade tänkt oss, dock inte riktigt så lättillgängligt som det var tänkt. Det skiljer sig även mellan Mac och Windows/Linux vilket inte var ursprungstanken.

\section{Arbetsplan. Beskriv hur arbetet kommer att delas upp mellan personerna i projektet. Gör en tidsplan som visar när olika delmoment i projektet ska vara klara.}

\noindent {\bf A:}

\noindent Adrian arbetar med det grundläggande grafikarbetet och lågnivåkod och algoritmer/optimisering. \\

\noindent Casper arbetar med användarvänlighet, javaspecifika funktioner och högnivåkod samt hjälper till med grafikarbetet. \\

\noindent Vi har redan fått en fullt fungerande prototyp färdig, och det mesta som är kvar att implementera är användargränssnittet samt avancerade extrafunktioner. \\

\noindent UI är högsta prioritet i nuläget, vi måste få ett fungerande och lättanvänt gränssnitt som inte är förvirrande. Sedan efter det gäller det att koppla ihop UIt med funktionerna i MandelbrotGenerator, så att användaren själv kan ändra på bilden. \\

\noindent Lite mindre prioriterade funktioner är att kunna spara bilder, samt att ändra på mandelbrotalgoritmen så att den genererar ett juliaset. \\

\noindent Funktioner som vi har i åtanke att implementera, men inte är säkra på ifall en implementation är helt trolig är bland annat användarspecificerade funktioner, dragging av bilder. \\
\\
\\
{\bf B:} \noindent Adrian arbetade med det grundläggande grafikarbetet och lågnivåkoden och algoritmer/optimisering. Jobbade även med design.

\noindent Casper arbetade med alla GUI-relaterade som layout med mer. Arbetade även på delar av det grundläggande grafikarbetet.
\\
\noindent Vi prioriterade i första hand UI:t för att få det så användarvänligt som möjligt och såg till att användaren själv kan ändra på en del inställningar så som färger och detaljnivå.

Vi såg även till att användaren kan spara bilder och generera juliaset.
\\
{\bf C:} På grund av tidsbegränsningen hann vi inte med att implementera "dragging" av bilder eller användardefinierade funktioner. i övrigt hann vi med det vi ville uppnå.

\section{Sammanfattning}
I detta projekt har vi lärt oss en hel del om att programmera tillsammans med andra, version control system samt projektplanering. Vi har utöver det lärt oss mycket om hur javax.swing fungerar på olika operativsystem, hur mandelbrot- och juliaset fungerar samt paralellisering i algoritmer. Det har varit ett par mycket stressfyllda veckor och i vissa fall blivit mycket frustrerande med viss problematisk kod, men vi känner att det har varit lärorikt ändå och vi har fått en liten inblick över hur det skulle kunna vara ute i arbetslivet.

Vi skulle eventuellt utveckla programmet med instruktioner och möjlighet till användarspecifika funktioner men i nuläget känner vi oss nöjda med det vi åstadkommit och har inga planer med att fortsätta utveckla programmet.
\end{document}
